\documentclass{article}
\usepackage[utf8]{inputenc}

\title{Image Classification}
\author{Jaehyun Kim, Alex Vobornov}
\date{August 2019}

\begin{document}

\maketitle

\section{Perceptron}
    \subsection{Extracting}
        \hspace*{10mm}In the classifier.py, the raw data from training data return a set of pixel features indicating whether each pixel in the provided datum is '0' if it is empty ,or '1' if it is '+' or '\#'. This works occurs at basicFeatureExtractorDigit and basicFeatureExtractorFace functions. For example, the feature will be \{(0,0):0,(0,1):0,(0,2):1...(28,28):.0\}. 
   
    \subsection{Perceptron Classifier}
        \hspace*{10mm}PerceptronClassifier is called by classifier. Arguments of 'legalLabels' and 'option.iterations' are used. legalLabels is type of list that include 0 to 9 to compare labels. option.iterations is the option with -i that how many repeat the training. The default iteration is 3. That means hidden layers are 3. \newline
        For each layer, we scan one instance at a time and find the label with the highest score. To get a high score, we set y as real label and y' is guess label. The guess label can get from this formula \ y' = arg max score(feature, y")\ . \newline
        To get y', guessed label, we use classify function in perceptron class. Multiply weight[label] and current data as datum. It means multiply each features of value and then return the sum of the values to vector which is type of list. 
        For example,  
        \begin{quote}
            vectors[l] = weights[l] * datum\\
            = $sum += weight[label] * datum[label]$\\
            = $sum += (a,b):z * (a',b'):z' = z * z'$\\
            = $((0,0):0 * (0',0'):0)+...+((28,28):0 * (28,28):0)$\\
            vector[l] = (0*0)+...+(0*0)\\
            vector = \{0:0, 1:144, 2:4, ... , 9:0\}
        \end{quote}
        After calculating all of the vectors, we choose the maximum number in the vector list and using argMax function, which is in util.py, we return the highest key of the highest value.
        
        And then, We compare y and y'. If y and y' are same then skip. If y and y' are not matching. It means that we guessed y' but we should have guessed y. The weight of y should have scored f higher and weight of y' should have scored f lower. \newline
        In our code, for example, we used 100 of training data to see how y and y' work. \newline
        \begin{quote}
        ('Starting iteration ', 0, '...')\\
        We have guessed 0 but should have guessed 5\\
        We have guessed 5 but should have guessed 0\\
        We have guessed 0 but should have guessed 4\\
        We have guessed 5 but should have guessed 1\\
        We have guessed 0 but should have guessed 9\\
        We have guessed 0 but should have guessed 2\\
        We have guessed 9 but should have guessed 1\\
        We have guessed 1 but should have guessed 3\\
        We have guessed 1 but should have guessed 4\\
        We have guessed 1 but should have guessed 3\\
        We have guessed 3 but should have guessed 5\\
        We have guessed 3 but should have guessed 6\\
        We have guessed 3 but should have guessed 1\\
        We have guessed 3 but should have guessed 7\\
        We have guessed 3 but should have guessed 2\\
        We have guessed 3 but should have guessed 8\\
        We have guessed 3 but should have guessed 6\\
        We have guessed 1 but should have guessed 9\\
        We have guessed 3 but should have guessed 4\\
        We have guessed 3 but should have guessed 0\\
        We have guessed 4 but should have guessed 9\\
        We have guessed 9 but should have guessed 1\\
        We have guessed 3 but should have guessed 2\\
        We have guessed 2 but should have guessed 4\\
        We have guessed 2 but should have guessed 3\\
        We have guessed 3 but should have guessed 2\\
        We have guessed 2 but should have guessed 7\\
        We have guessed 2 but should have guessed 3\\
        We have guessed 3 but should have guessed 8\\
        We have guessed 3 but should have guessed 6\\
        We have guessed 3 but should have guessed 9\\
        We have guessed 3 but should have guessed 0\\
        We have guessed 3 but should have guessed 5\\
        We have guessed 3 but should have guessed 6\\
        We have guessed 2 but should have guessed 0\\
        We have guessed 2 but should have guessed 7\\
        We have guessed 3 but should have guessed 1\\
        We have guessed 0 but should have guessed 8\\
        We have guessed 3 but should have guessed 7\\
        We have guessed 0 but should have guessed 9\\
        We have guessed 3 but should have guessed 8\\
        We have guessed 0 but should have guessed 5\\
        We have guessed 3 but should have guessed 9\\
        We have guessed 0 but should have guessed 7\\
        We have guessed 9 but should have guessed 4\\
        We have guessed 9 but should have guessed 8\\
        We have guessed 9 but should have guessed 4\\
        We have guessed 8 but should have guessed 1\\
        We have guessed 8 but should have guessed 4\\
        We have guessed 4 but should have guessed 6\\
        We have guessed 6 but should have guessed 4\\
        We have guessed 4 but should have guessed 5\\
        We have guessed 8 but should have guessed 1\\
        We have guessed 2 but should have guessed 0\\
        We have guessed 4 but should have guessed 0\\
        We have guessed 0 but should have guessed 1\\
        We have guessed 4 but should have guessed 7\\
        We have guessed 4 but should have guessed 6\\
        We have guessed 0 but should have guessed 3\\
        We have guessed 1 but should have guessed 2\\
        We have guessed 4 but should have guessed 7\\
        We have guessed 0 but should have guessed 9\\
        We have guessed 0 but should have guessed 2\\
        We have guessed 2 but should have guessed 7\\
        We have guessed 7 but should have guessed 9\\
        We have guessed 9 but should have guessed 4\\
        We have guessed 9 but should have guessed 8\\
        We have guessed 9 but should have guessed 7\\
        ('Starting iteration ', 1, '...')\\
        We have guessed 8 but should have guessed 5\\
        We have guessed 4 but should have guessed 3\\
        We have guessed 4 but should have guessed 5\\
        We have guessed 8 but should have guessed 1\\
        We have guessed 7 but should have guessed 1\\
        We have guessed 7 but should have guessed 4\\
        We have guessed 1 but should have guessed 8\\
        We have guessed 7 but should have guessed 9\\
        We have guessed 8 but should have guessed 0\\
        We have guessed 1 but should have guessed 5\\
        We have guessed 0 but should have guessed 6\\
        We have guessed 8 but should have guessed 5\\
        We have guessed 0 but should have guessed 3\\
        We have guessed 9 but should have guessed 4\\
        We have guessed 4 but should have guessed 9\\
        We have guessed 9 but should have guessed 8\\
        We have guessed 8 but should have guessed 1\\
        We have guessed 9 but should have guessed 7\\
        We have guessed 1 but should have guessed 2\\
        We have guessed 0 but should have guessed 9\\
        We have guessed 9 but should have guessed 6\\
        We have guessed 9 but should have guessed 3\\
        We have guessed 9 but should have guessed 7\\
        ('Starting iteration ', 2, '...')\\
        We have guessed 3 but should have guessed 5\\
        We have guessed 9 but should have guessed 2\\
        We have guessed 9 but should have guessed 1\\
        We have guessed 2 but should have guessed 0\\
        \end{quote}
            

        
        In this case, we updated the weights as \ $w^y = w^y + f$ \ and \ $w^y' = w^y' - f$.\ In the code, we uses util.py methods which are $\_\_$radd$\_\_$ and $\_\_$sub$\_\_$ to increment/decrement counters.
        \begin{quote}
            $self.weights[y].$\_\_$radd$\_\_$(trainingData[i])$\\
            $self.weights[yPrime].$\_\_$sub$\_\_$(trainingData[i])$
        \end{quote}
            
    \subsection{Training data}
        \begin{table}[]
            \centering
            \begin{tabular}{c|c}
                 &  \\
                 & 
            \end{tabular}
            \caption{Caption}
            \label{tab:my_label}
        \end{table}
    \subsection{analysis}
        To analyze for our accuracy, we calculate percentage of validate data and test data. \newline
        For valid data with 100 trained, it shows us 69 correct out of 100 (69.0\%). And for test data with 100 trained it shows us 62 correct out of 100 (62.0\%). \newline
        


\section{Naive Bayes}

\end{document}
